\chapter{Выполнение лабораторной работы}


\section{Цель работы}

Целью лабораторной работы является отработка навыков использования программы \textbf{Microsoft Project} для оптимизации временных и финансовых показателей проекта.


\section{Краткое описание проекта}

Команда разработчиков из 16 человек занимается созданием карты города на основе собственного модуля отображения. Проект должен быть завершен в течение 6 месяцев. Бюджет проекта: 50 000 рублей.


\section{Способы устранения перегрузок}

Для устранения прегрузок могут использованы следующие способы:

\begin{itemize}
    \item изменить календарь работы ресурса;
    \item назначить ресурс на неполный рабочий день;
    \item изменить профиль назначения ресурса;
    \item изменить ставку оплаты ресурса;
    \item добавить ресурсу времся задержки;
    \item разбить задачу на этапы и перекрыть по времени их выполнение;
    \item применить автоматическое выравнивание.
\end{itemize}


\section{Данные из лабораторной работы №2}

\subsection{Информация о ресурсах}

На рисунке \ref{img:lab2-res} представлен список ресурсов проекта.

\imgs{lab2-res}{h!}{0.3}{Лист ресурсов}


\subsection{Информация о перегрузках}

На рисунке \ref{img:lab2-error} представлен визуальный оптимизатор ресурсов, на котором видно, что перегрузки возникают у системного аналитика, художника дизайнера и технического писателя. Перегрузки возникают по причине того, что эти ресурсы выполняются параллельно в нескольких задачах.

\imgs{lab2-error}{h!}{0.3}{Визуальный оптимизатор ресурсов с перегрузками}


\section{Задание 1}

\subsection{Ликвидировать перегрузки в проекте}

Для ликвидации перегрузок в проекте будет использован способ автоматического выравимания с превышением доступности по дням. Настройки выравнивания представлены на рисунке \ref{img:task1-improv}. Получившийся результат выравнивания виден на диаграмме Ганта с выравниванием (рисунок \ref{img:task1-res}). При этом длительность проекта увеличилась на 4 дня и окончание установлено на 19 сентября 2023 года. Также на рисунке \ref{img:task1-res-resource} показано, что ресурсы не имеют перегрузки. Автоматическое выравнивание решило перегрузки засчет переноса задач, которые не влияют на критический путь. Перегрузки были решены:

\begin{itemize}
    \item для задач системного аналитика и художника-дизайнера перенесены задачи на более поздний срок;
    \item для технического писателя задача была задержана на 4 дня, так как обе его параллельные задачи являются задачами критического пути.
\end{itemize}

\imgs{task1-improv}{h!}{0.5}{Настройки автоматического выравнивания ресурсов}
\imgs{task1-res}{h!}{0.4}{Результат выравнивания на диаграмме Ганта}
\imgs{task1-res-resource}{h!}{0.3}{Результат выравнивания на листе ресурсов}



\section{Задание 2}

\subsection{Добавить совещание по средам длительностью в 1 час}

На рисунке \ref{img:task2-meeting-settings} представлены параметры добавления повторяющейся задачи. 

\imgs{task2-meeting-settings}{h!}{0.4}{Добавление совещания в задачи проекта}


\subsection{Добавить спициалистов в совещание}

Задаче <<Совещание>> добавлены все сотрудники (ресурсы), кроме наборщиков данных и программистов 1-4 (рисунок \ref{img:task2-members}). Результат добавления задачи <<Совещание>> и выдаче ей русрсов представлен на рисунке \ref{img:task2-meeting-res}.

\imgs{task2-members}{h!}{0.5}{Добавление ресурсов совещанию}
\imgs{task2-meeting-res}{h!}{0.5}{Совещание в проекте}


\subsection{Устранить возникшую перегрузку}

В результате добавления совещания для ряда реурсов возникли перегрузки (рисунки \ref{img:task2-error-list}-\ref{img:task2-error-visual}). Для их устранения проведем еще одно автоматическое выравнивания с превышинием доступности по неделям. Результат устранения перегрузок приведен на рисунке \ref{img:task2-error-corrected}.

\imgs{task2-error-list}{h!}{0.3}{Перегрузка ресурсов на листе}
\imgs{task2-error-visual}{h!}{0.3}{Перегрузка ресурсов на визуальном оптимизаторе}
\imgs{task2-error-corrected}{h!}{0.3}{Результат устранения перегрузок}


\subsection{Оптимизация бюджета}

После добавления совещания бюджет был увеличен и стал превышать допустимый объем средств в 50 000 рублей (рисунок \ref{img:task2-high-budget}). Возникло превышение бюджета из-за того, что участие сотрудника на совещании учитывается как его использование. Но поскольку сотрудник не занимается на совещании своими прямыми обязанностями, необходимо добавить новый план оплаты и установить его сотрудникам, участвующим в совещании. 

Для этого в листе ресурсов для каждого используемого сотрудника добавить новый план оплаты, где затраты на использование будут установлены на отметке 0 рублей (рисунок \ref{img:task2-new-payplan}). После этого данный план оплаты должен быть выбран для каждого сотрудника в совещании. Это можно сделать во вкладке <<Вид>>, в которой нужно нажать кнопку <<Использование задач>>. Появится список задач с распределением ресурсов по ним, в котором нужно добавить новый столбец -- <<Таблица норм затрат>> и проставить всем сотрудникам, используемым на совещании, план оплаты <<B>> (рисунок \ref{img:task2-set-payplan}). После этого бюджет проекта сократится и станет соответствовать имеющимся средствам, что видно на рисунке \ref{img:task2-payplan-res}.

\imgs{task2-high-budget}{h!}{0.4}{Превышение бюджета}
\imgs{task2-new-payplan}{h!}{0.5}{Новый план оплаты}
\imgs{task2-set-payplan}{h!}{0.4}{Установка нового плана оплаты}
\imgs{task2-payplan-res}{h!}{0.4}{Оптимизированный бюджет}


\section{Задание 3}

\subsection{Анализ задач критического пути}

На рисунке \ref{img:task3-critical-way} показан критический путь проекта. Из данного пути видно, что самыми долгими явялются задачи, связанные с программирование. Для сокращения сроков выполнения нужно провести оптимизацию критического пути.

\imgs{task3-critical-way}{h!}{0.3}{Критический путь проекта}


\subsection{Оптимизация критического пути}

Поскольку самыми долгими явялются задачи, связанные с программированием, проанализируем их в визуальном оптимизаторе. Как видно из рисунка \ref{img:task3-visual-non-optim}, работа между программистами 1-4 распределена неравномерно. Для оптимизации назначим дополнительных программистов на задачи 7, 10, 14, 15, 16, 26 и проведем ликивадицю перегрузок (также были убраны совещания после 7 июля -- новой даты окончания проекта). По результату этих действий удалось сократить длительность проекта и изменить дату его окончания с 22.9.23 до 18.7.23, что видно на рисунке \ref{img:task3-task-optim}. Также на рисунке \ref{img:task3-visual-optim} видно, что теперь программисты работают равномерно по всей длительность проекта.

\imgs{task3-visual-non-optim}{h!}{0.4}{Визуальный оптимизатор ресурсов до оптимизации критического пути}
\imgs{task3-task-optim}{h!}{0.3}{Проект после оптимизации задач}
\imgs{task3-visual-optim}{h!}{0.4}{Визуальный оптимизатор ресурсов после оптимизации критического пути}


\clearpage

\subsection{Cравнение с результатами лабораторной работы №2}

Сравнение затрат:

\begin{itemize}
    \item затраты в лабораторной работе №2 (рисунок \ref{img:task3-lab2-money});
    \item затраты в лабораторной работе №3 (рисунок \ref{img:task3-lab3-money}).
\end{itemize}


Сравнение трудозатрат:

\begin{itemize}
    \item трудозатраты в лабораторной работе №2 (рисунок \ref{img:task3-lab2-work});
    \item трдуозатраты в лабораторной работе №3 (рисунок \ref{img:task3-lab3-work}).
\end{itemize}

\imgs{task3-lab2-money}{h!}{0.5}{Затраты в лабораторной работе №2}
\imgs{task3-lab3-money}{h!}{0.5}{Затраты в лабораторной работе №3}
\imgs{task3-lab2-work}{h!}{0.5}{Трудозатраты в лабораторной работе №2}
\imgs{task3-lab3-work}{h!}{0.5}{Трдуозатраты в лабораторной работе №3}


\section{Сохранение базового плана проекта}

На рисунке \ref{img:task3-saveplan} предаставлен сохраняемый план проекта.

\imgs{task3-saveplan}{h!}{0.3}{План проекта}


\section*{Вывод}

\begin{enumerate}
    \item По результату оптимизации проекта, изменен срок оночания проекта с 22.9.23 до 18.7.23.
    \item Затраты на группы программистов и сервер сократились по 1 проценту, а затраты на группы документации и анализа увеличились на 1 процент.
    \item Трудозатраты на группу сервера сократились на 2 процента, а на группы дизайна и анализа увеличились по 1 проценту.
\end{enumerate}

