\chapter{Выполнение лабораторной работы}


\section{Цель работы}

Целью лабораторной работы является знакомство с возможностями программы \textbf{Microsoft Project} по контролю ходом реализации проекта.


\section{Краткое описание проекта}

Команда разработчиков из 16 человек занимается созданием карты города на основе собственного модуля отображения. Проект должен быть завершен в течение 6 месяцев. Бюджет проекта: 50 000 рублей.


\section{Данные из лабораторной работы №3}

На рисунках \ref{img:lab3-time}-\ref{img:lab3-res} представлено состояние проекта на момент окончания выполнения лабораторной работы №3.

\imgs{lab3-time}{h!}{0.3}{Диаграмма Ганта из ЛР №3}
\imgs{lab3-res}{h!}{0.3}{Лист ресурсов из ЛР №3}


\section{Выполнение заданий}

В первую очередь, установим дату отчета на 12 мая.

\imgs{report-date}{h!}{0.5}{Установка даты отчета}


\subsection{Задание 1}

\textbf{Задание:} С 6 марта на 10 дней заболел 3d-аниматор. В это время его обязанности выполнял художник-дизайнер, совмещая их со своими работами по проекту из расчета 70\% доступности по своим задачам и 30\% --- по задачам 3d-аниматора. В этот период зарплата художника–дизайнера увеличилась на 10\%.

Установим в ресурсе, что 3D-аниматор был не готов работать в промежутке с 6 по 16 марта (рисунок \ref{img:task1-3d-changed}). При этом изменим загруженность ресурсов в задачах в этот промежуток времени. Таким образом, загруженность художника-дизайнера в задаче <<Разработка 2D графических элементов>> (рисунок \ref{img:task1-work1})сократилась до 70\%, при этом он стал приничать 30\% участия в задачах <<Создание заставки>> (рисунок \ref{img:task1-work2}) и <<Разработка 3D графических элементов>> (рисунок \ref{img:task1-work3}) (при этом возникли перегрузки (рисунок \ref{img:task1-over}), которые были решены автоматическим выравниванием; при этом дата окончания проекта не была сдвинута). Также зарплата художника-дизайнера дизайнера на этот период была повышена на 10\%, что видно на рисунке \ref{img:task1-more-money}. Стоимость проекта при этом выросла на 73 рубля.

\imgs{task1-3d-changed}{h!}{0.4}{Доступность 3D-аниматора}
\imgs{task1-work1}{h!}{0.4}{Ресурсы в задаче <<Разработка 2D графических элементов>>}
\imgs{task1-work2}{h!}{0.4}{Ресурсы в задаче <<Создание заставки>>}
\imgs{task1-work3}{h!}{0.4}{Ресурсы в задаче <<Разработка 3D графических элементов>>}
\imgs{task1-over}{h!}{0.4}{Перегрузки в задании 1}
\imgs{task1-more-money}{h!}{0.4}{Увеличение ставки художника-дизайнера}


\clearpage
\subsection{Задание 2}

\textbf{Задание:} Задача №5 фактически началась 6.03.

Выставим фактическую дату начала задачи №5 <<Разработка 3D графических элементов>> на 6 марта (рисунок \ref{img:task2-new-date}). При этом возникли перегрузки (\ref{img:task2-over}), которые были исправлены автоматическим выравниванием. При этом дата окончания проекта не была сдвинута.

\imgs{task2-new-date}{h!}{0.4}{Установка фактической даты начала задачи №5}
\imgs{task2-over}{h!}{0.4}{Перегрузки в задании 2}


\clearpage
\subsection{Задание 3}

\textbf{Задание:} Фактическая длительность задач №9-10 оказалась на 10\% больше.

Выставим новую остаточную длительность задачам №9 <<Анализ и построение структуры базы объектов>> (рисунок \ref{img:task3-work9}) и №10 <<Программирование средств обработки базы объектов>> (рисунок \ref{img:task3-work10}), которая на 10\% больше текущей. При этом возникли перегрузки (рисунок \ref{img:task3-over}), которые были решены автоматическим выравниванием (дата окончания проекта не изменилась).

\imgs{task3-work9}{h!}{0.4}{Обновленная длительность для задачи №9}
\imgs{task3-work10}{h!}{0.4}{Обновленная длительность для задачи №10}
\imgs{task3-over}{h!}{0.4}{Перегрузки в задании 3}


\clearpage
\subsection{Задание 4}

\textbf{Задание:} 3 апреля для задачи <<Тестирование модели ядра>> купили специализированное ПО стоимостью 800 рублей и еще 200 рублей понадобилось на его установку.

Добавим новый \textit{материальный} ресурс <<Спец ПО>>, которе купили 3 апреля (рисунок \ref{img:task4-new-res}). Данный ресурс был куплен для задачи <<Тестирование модели ядра>>, что видно на рисунке \ref{img:task4-add-res}. Стоимость проекта выросла на 1000 руб и составляет 49 652 рубля.

\imgs{task4-new-res}{h!}{0.4}{Новый материальный ресурс}
\imgs{task4-add-res}{h!}{0.4}{Выдача ресурса задаче <<Тестирование модели ядра>>}


\clearpage
\subsection{Задание 5}

\textbf{Задание:} С 1 апреля на 5\% была увеличена зарплата всех программистов, кроме ведущего.

Всем программистам была увеличена зарпалата на 5\%, что видно на рисунке \ref{img:task5-more-money}. При этом стоимость проекта возросла до 50 393 рублей.

\imgs{task5-more-money}{h!}{0.4}{Изменение заработной платы программистов с 1 апреля}


\clearpage
\subsection{Задание 6}

\textbf{Задание:} С 1 апреля в совещании стали участвовать только те специалисты, чьи задачи реализуются на текущей неделе. От группы программистов участвует по-прежнему ведущий программист.

Начиная с 1 апреля, были проставлены на совещание только те сотрудники, чьи задачи выполняются на текущей неделе (пример на рисунке \ref{img:task6-meeting}). При этом траты на совещания сократились с 1 220 рублей до 610 рублей, а стоимость проекта с 50 393 рублей --- до 49 774 рублей.

\imgs{task6-meeting}{h!}{0.4}{Пример выставления сотрудников на совещание}


\clearpage
\section*{Вывод}

Для просмотра отклонений в проекте покажем линию прогресса (рисунок \ref{img:final-enable-line}). Результат виден на рисунке \ref{img:final-line-res}.

Также на рисунке \ref{img:final-comparsion} предаставлено сравнение получившихся деталей проекта с базовыми. Так, длительность проекта и дата его окончания сдвинулась на один день (с 18 июля на 19 июля), при этом изменились затраты на проект --- с 48 560 рублей до 49 774 рублей.

Таким образом, проект не вышел ни за предел по затратам, ни по длительности выполнения, значит оптимизация не требуется.

\imgs{final-enable-line}{h!}{0.4}{Показать линию прогресса}
\imgs{final-line-res}{h!}{0.3}{Линия прогресса на диаграмме Ганта}
\imgs{final-comparsion}{h!}{0.3}{Сравнение базовых параметров проекта с текущими}
