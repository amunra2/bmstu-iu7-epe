\chapter{Выполнение лабораторной работы}


\section{Цель работы}

Целью лабораторной работы является ознакомление с существующими методиками предварительной оценки параметров программного проекта и практическая оценка затрат на примере методики \textbf{COCOMO} (COnstructive COst MOdel — конструктивная модель стоимости).


\section{Модель оценки стоимости COCOMO}

\textbf{COnstructive COst MOdel} -- алгоритмическая модель оценки стоимости разработки программного обеспечения. Она использует простую формулу регрессии с параметрами, определенными из данных, собранных по ряду проектов.

\begin{equation}
    \text{Трудозатраты} = \text{С1} \cdot \text{EAF} \cdot (\text{Размер})^{\text{p1}}
\end{equation}

\begin{equation}
    \text{Время} = \text{С2} \cdot (\text{Трудозатраты})^{\text{p2}}
\end{equation}

\noindent где:

\begin{itemize}
    \item \textit{трудозатраты} -- количество человеко-месяцев;
    \item \textit{С1} — масштабирующий коэффициент;
    \item \textit{EAF} — уточняющий фактор, характеризующий предметную область, персонал, среду и инструментарий, используемый для создания рабочих продуктов процесса;
    \item \textit{размер} — размер конечного продукта (кода, созданного человеком), измеряемый в исходных инструкциях (DSI, delivered source instructions), которые необходимы для реализации требуемой функциональной возможности;
    \item \textit{P1} — показатель степени, характеризующий экономию при больших масштабах, присущую тому процессу, который используется для создания конечного продукта; в частности, способность процесса избегать непроизводительных видов деятельности (доработок, бюрократических проволочек, накладных расходов на взаимодействие);
    \item \textit{время} — общее количество месяцев;
    \item \textit{С2} — масштабирующий коэффициент для сроков исполнения;
    \item \textit{Р2} — показатель степени, который характеризует инерцию и распараллеливание, присущие управлению разработкой ПО.
\end{itemize}


\section{Задание 1}

\subsection{Условие}

Исследовать влияние атрибутов программного продукта (\textbf{RELY}, \textbf{DATA} и \textbf{CPLX}) на трудоемкость (\textbf{РМ}) и время разработки (\textbf{ТМ}) для модели \textbf{COCOMO} и промежуточного типа проекта. Для этого получить значения PM и ТМ для одного и того же значения размера программного кода (\textbf{SIZE}), изменяя значения указанных драйверов от очень низких до очень высоких. Сначала провести анализ при отсутствии ограничений на сроки разработки, выбрав номинальное значение параметра \textbf{SCED}. 

Какой из трех указанных драйверов затрат оказывает большее влияние на сроки реализации проекта и объем работ? Проанализировать, как изменятся значения \textbf{PM} и \textbf{ТМ} при наличии более жестких ограничений на сроки разработки (драйвер \textbf{SCED} изменяется от высокого до очень высокого). Результаты исследований оформить графически и сделать соответствующие выводы.


\subsection{Результаты}

На рисунках \ref{img:task1-normal}-\ref{img:task1-high-high} приведены результаты графических исследований атрибутов (\textbf{RELY}, \textbf{DATA} и \textbf{CPLX}) при различных значениях атрибута \textbf{SCED}.

\imgs{task1-normal}{t!}{0.4}{Уровень: номинальный}
\imgs{task1-high}{t!}{0.4}{Уровень: высокий}
\imgs{task1-high-high}{t!}{0.4}{Уровень: очень высокий}

\clearpage
\subsection{Выводы}

\begin{enumerate}
    \item При увеличении уровней исследуемых атрибутов растут трудозатраты и время, так как требования к проекту повышаются.
    \item При номинальном значении драйвера \textbf{SCED} наибольшее влияние на сроки реализации проекта и объем работ оказывает драйвер \textbf{RELY} (требуемая надежность). Это видно, когда его уровень достигает значения <<Очень высокий>> -- трудозатраты и время в этом случае сильно увеличиваются, относительно остальных исследуемых драйверов. При этом атрибут \textbf{DATA} (размер базы данных) меньше всех драйверов влияет на увеличение объема трудозатрат и сроки проекта.
    \item Более строгие ограничения на сроки разработки не сильно влияют на трудозатраты и время. Так, при уровне <<Высокий>> трудозатраты повышаются лишь на 4\%, а время -- на 1.5\%. А для уровня <<Очень высокий>> -- трудозатраты на 6\%, а время -- на 2\% (относительно <<Высокого>> уровня).
\end{enumerate}


\section{Задание 2}

\subsection{Условие}

Компания разрабатывает программную систему управления воздушным движением. Программа обрабатывает сигналы радара и ответчика и преобразовывает их в цифровые данные, позволяющие авиадиспетчерам назначать курсы, высоту и скорость полетов. Разработка ведется командой \textit{высококвалифицированных специалистов} в рамках государственного контракта. Предполагаемый размер разрабатываемой системы \textit{430 000} строк кода. Система имеет \textit{высокие требования по надежности}, \textit{жесткие ограничения на время выполнения} и \textit{сроки разработки}. Используется \textit{промежуточный} режим модели.


\subsection{Результат}

По условию имеем:

\begin{itemize}
    \item драйвера персонала -- максимально возможный уровень;
    \item драйвер \textbf{RELY} (требуемая надежность) -- уровень <<Высоко>>;
    \item драйвер \textbf{TIME} (ограничение времени выполнения) -- уровень <<Очень высоко>>;
    \item драйвер \textbf{SCED} (требуемые сроки разработки) -- уровень <<Очень высоко>>;
    \item размер (\textbf{KLOC}) -- 430;
    \item режим -- промежуточный.
\end{itemize}

На рисунке \ref{img:task2-result} представлен результат рассчетов проекта с данными значениями.

\imgs{task2-result}{h!}{0.3}{Результат рассчетов проекта}


\subsection{Выводы}

\begin{enumerate}
    \item Трудозатраты:
        \begin{itemize}
            \item без учета планирования -- 1765.73 человеко-месяцев;
            \item с учетом планирования -- 1906.99 человеко-месяцев.
        \end{itemize}
    \item Время:
        \begin{itemize}
            \item без учета планирования -- 34.23 месяцев;
            \item с учетом планирования -- 46.55 месяцев.
        \end{itemize}
    \item При средней зарплате в 150 тысяч рублей -- бюджет проекта составит 264 миллиона рублей.
    \item Наибольшие затраты на <<Программирование>> -- 116 миллионов рублей.
    \item Наибольшее число сотрудников будет задействовано на этапе <<Кодирование и тестирование отдельных модулей>> -- 75 человек и на этапе <<Детальное проектирование>> -- 72 человека.
\end{enumerate}


\section{Вывод}

При выполнении лабораторной работы был разработан программный инструмент для оценки проекта по методике \textbf{COCOMO}. Данная методика подходит дл предварительной оценки длительности и затрат проекта на каждом из его этапов. При этом данная оценка является грубой, поэтому следует использовать и другие методики для более точных значений.
