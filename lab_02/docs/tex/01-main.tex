\chapter{Выполнение лабораторной работы}


\section{Цель работы}

Целью лабораторной работы №2 является освоение возможностей программы \textbf{Microsoft Project} для работы с ресурсами.


\section{Тренировочное задание}

Добавить в тренировочное задание из лабораторной работы №1 информацию о ресурсах (стандартная ставка 250 руб/день) и определить стоимость проекта. Также добавить фиксированную выплату в \texttt{5\ 000} руб и с понедельника второй недели реализации проекта для работ A, B, C добавить ресурс со ставкой в \texttt{1\ 000} руб/нед. Ресурсы представлены в таблице \ref{tbl:resource}.

% Настройка выравнивание колонки по центру
\newcolumntype{P}[1]{>{\centering\arraybackslash}p{#1}}

\begin{center}
    \captionsetup{justification=raggedleft,singlelinecheck=off}
    \begin{longtable}[c]{|P{5cm}|P{5cm}|}
    \caption{Ресурсы проекта\label{tbl:resource}}
    \\ \hline
        \textbf{Название работы} & \textbf{Количество исполнителей (чел.)}
    \\ \hline
        Работа A & 1
    \\ \hline
        Работа B & 4
    \\ \hline
        Работа C & 5
    \\ \hline
        Работа D & 2
    \\ \hline
        Работа E & 3
    \\ \hline
        Работа F & 3
    \\ \hline
        Работа J & 5
    \\ \hline
        Работа H & 5
    \\ \hline
        Работа I & 2
    \\ \hline
        Работа J & 6
    \\ \hline
\end{longtable}
\end{center}


\subsection{Выполнение задания}

\begin{enumerate}
    \item Заполнен лист ресурсов для задания (рисунок \ref{img:test-list}). Также на рисунке \ref{img:test-extra-res} представлены настройки для добавления дополнительного ресурса для задач A, B, C.
    \imgs{test-list}{h!}{0.4}{Лист ресурсов}
    \imgs{test-extra-res}{h!}{0.4}{Дополнительный ресурс}

    \item Ресурсы распределены по задачам проекта в соответствии с заданием. При этом получено, что суммарная стоимость проекта равна \texttt{119 750} руб (рисунок \ref{img:test-task}).
    \imgs{test-task}{h!}{0.5}{Список задач проекта}

    \item Как видно из полученного результата (рисунок \ref{img:test-result}), в некоторых задачах проекта возникли перегрузки. Это связано с нехваткой ресурсов для их выполнения.
    \imgs{test-result}{h!}{0.5}{Визуальный оптимизатор ресурсов проекта}
\end{enumerate}


\section{Краткое описание проекта}

Команда разработчиков из 16 человек занимается созданием карты города на основе собственного модуля отображения. Проект должен быть завершен в течение 6 месяцев. Бюджет проекта: 50 000 рублей.


\section{Задание 1}

Заполнить ресурсный лист проекта.

\subsection{Выполнение задания}

На рисунке \ref{img:task1} представлен заполненный в соответствии с заданием лист ресурсов.

\imgs{task1}{h!}{0.35}{Список ресурсов}


\section{Задание 2}

Назаначить ресурсы из задания 1 задачам. Задать задачам 2, 8, 12 фиксированные затраты в размере 1000 руб. Также арендовать для задачи 8 дополнительный сервер для за 2 руб/час.

\subsection{Выполнение задания}

Аренда сервера добавлена в лист ресурсов со ставкой 2 руб/час и с 24-часовым календарем, так как сервер работает круглосуточно (рисунок \ref{img:tsk2-server}). Также на рисунке \ref{img:task2-result} представлено распределение ресурсов по задачам, а на рисунке \ref{img:task2-gant} -- диаграмма Ганта проекта. При этом возникли перегрузки из-за нехватки ресурсов, что видно на рисунке \ref{img:task2-error}.

\imgs{task2-server}{h!}{0.3}{Дополнительный ресурс проекта}
\imgs{task2-result}{h!}{0.3}{Распределение ресурсов по задачам}
\imgs{task2-gant}{h!}{0.4}{Диаграмма Ганта}
\imgs{task2-error}{h!}{0.3}{Визуальный опитимизатор проекта}


\section{Задание 3}

Провести структуризацию затрат по группам ресурсов и представить в графическом виде. Также сделать для трудозатрат.

\subsection{Выполнение задания}

На рисунке \ref{img:task3-struct} представлен структуризованный список ресурсов по группам. Также на рисунке \ref{img:task3-money} представлена круговая диграмма затрат по группам ресурсов, а на рисунке \ref{img:task3-work} -- диаграмма трудозатрат по группам ресурсов.

\imgs{task3-struct}{h!}{0.3}{Структурирозванный список ресурсов по группам}
\imgs{task3-money}{h!}{0.5}{Круговая диаграмма затрат по группам ресурсов}
\imgs{task3-work}{h!}{0.5}{Круговая диаграмма трудозатрат по группам ресурсов}


\newpage

\section*{Вывод}

\begin{enumerate}
    \item В проекте возникли перезгрузки для некоторых исполнителей, так как работу им приходится выполнять параллельно.
    \item Затраты на проект равны 48 094 руб. Так как бюджет проекта равен 50 000 руб, то его хватит для покрытия всех расходов.
    \item Трудозатраты напроект составят 9 377 часов.
    \item Программирование проекта займет 50 процентов денежных затрат, при этом по трудозатратам будет занята лишь треть.
    \item Ввод данных в проекте занимает лишь 11 процентов бюджета, хотя по трудозатратам равен 26 процентам (что практически столько же, сколько трудозатрат будет использовано на программирование).
    \item Аренда сервера занимает 10 процентов бюджета, что весьма дорого.
\end{enumerate}
