\chapter{Выполнение лабораторной работы}


\section{Цель работы}

Целью лабораторной работы является освоение возможностей программы \textbf{Microsoft Project} по управлению финансовыми потоками на основе анализа затрат.


\section{Краткое описание проекта}

Команда разработчиков из 16 человек занимается созданием карты города на основе собственного модуля отображения. Проект должен быть завершен в течение 6 месяцев. Бюджет проекта: 50 000 рублей.


\section{Выполнение заданий}

Дата отчета установлена на 12 мая (в соответствии с ЛР №4).

\imgs{report-date}{t!}{0.5}{Установка даты отчета}


\subsection{Задание 1}

\textbf{Задание:} Таблицы освоенного объема.

Просмотрим таблицу освоенного объема. Чтобы ее открыть перейдем во вкладку <<Вид>> и выберем в ней <<Таблицы>> -> <<Другие таблицы>> -> <<Освоенный объем>> (рисунок \ref{img:task1-table}).

Данная таблица содержит следующие поля:

\begin{itemize}
    \item \textbf{запланированный объем (3О)} --- это средства, которые были затрачены на выполнение задачи в период с начала проекта до выбранной даты отчета, если бы задача точно соответствовала графику и смете;
    \item \textbf{освоенный объем (OО)} --- это средства, которые были бы затрачены на выполнение задачи с самого начала проекта до выбранной даты отчета, если бы фактически выполненная работа оплачивалась согласно смете, т.е. это фактическое количество рабочих часов, оплачиваемых по сметным ставкам;
    \item \textbf{фактические затраты (Ф3)} --- средства, фактически потраченные на задачи в период с начала проекта до выбранной даты отчета.
    \item \textbf{отклонение от календарного плана (ОКП)} --- позволяет вычислить несоответствие сметы, которое вызвано различием между плановым и фактическим объёмом работы, если это величина меньше 0, то проект опаздывает.
    \item \textbf{отклонение по стоимости (ОПС)} --- сравнивает сметную и фактическую стоимость выполненной работы и позволяет выделить несоответствие сметы, вызванные разницей стоимости ресурсов, если эта величина меньше нуля, то проект вышел за пределы сметы.
    \item \textbf{предварительная оценка по завершении (ПОПЗ)} --- отображает ожидаемые общие затраты для задачи, расчет которых основан на предположении, что оставшаяся часть работы будет выполнена в точном соответствии со сметой. (прогноз по завершении)
    \item \textbf{затраты по базовому плану (БПЗ)} --- показывают фиксированные затраты и стоимость ресурсов согласно базовому плану.
    \item \textbf{отклонение по завершению (ОПЗ)} --- разность между БПЗ и ПОПЗ, если эта величина отрицательна, то наблюдается перерасход средств.
\end{itemize}

Как видно из полученной таблицы, на дату отчетного периода освоенный объем (25 466 руб) меньше, чем запланированный (26 445 руб). Это достигается за счет того, что фактические траты на <<Создание рабочей версии ядра>> меньше запланированных. При этом фактические траты на <<Создание интерфейса>> выше, чем запланированные.

По полученным показателям можно сделать следующий вывод:

\begin{itemize}
    \item ОКП < 0 --- означает, что проект опаздывает;
    \item ОПС > 0 --- означает, что проект укладывается в смету;
    \item ОПЗ > 0 --- означает, что отсутствует перерасход средств.
\end{itemize}

\imgs{task1-table}{t!}{0.3}{Таблица освоенного объема}


\clearpage
\subsection{Задание 2}

\textbf{Задание:} Работа с отчетами

Благодаря отчету (<<Отчет>> -> <<Наглядные отчеты>> -> <<Отчет о бюджетной стоимости>>), определим в какое время будет испытываться наибольшая потребность в деньгах. Как видно из рисунка \ref{img:task2-most-money}, наибольшее количество денег понадобится на 6-ой неделе проекта.

\imgs{task2-most-money}{h!}{0.4}{Затраты по неделям}

Также проанализируем превышение затрат. На рисунке \ref{img:task2-high-task} видно превышение стоимости по задачам. Наиболее заметные изменения относительно базового плана наблюдаются в:

\begin{itemize}
    \item превышение для <<Создание ядра GIS>>, так как было добавлено специализированное ПО;
    \item уменьшение для <<Совещания>>, так как с определенной даты в нем стали участвовать только те сотрудники, чьи задачи выполнялись на текущуй неделе.
\end{itemize}

Также на рисунке \ref{img:task2-high-res} видно превышение стоимости по ресурсам. Наиболее заметны следующие изменения:

\begin{itemize}
    \item <<Художник-дизайнер>> заработал больше, а <<Мультимедиа-корреспондент>>, наоборот, меньше. Это связано с тем, что последний болел некоторое и его заменял <<Художник-дизайнер>>, который имел из-за этого повышенную ставку на данный период;
    \item <<Спец ПО>> превышено на всю свою стоимость, так как раньше не входило в проект.
\end{itemize}

\imgs{task2-high-task}{h!}{0.4}{Превышение стоимости по задачам}
\imgs{task2-high-res}{h!}{0.4}{Превышение стоимости по ресурсам}


\clearpage
\subsection{Задание 3}

\textbf{Задание:} декомпозиция проекта.

Каскадная модель жизненного цикла имеет следующий вид:

\begin{itemize}
    \item анализ;
    \item проектирование;
    \item разработка;
    \item тестирование;
    \item техническая поддержка.
\end{itemize}

В соответствии с данной моделью была проведена декомпозиция проекта на основе ЛР2. Результат приведен на рисунке \ref{img:task3-decomp}. Также на рисунке \ref{img:task3-lab3} представлен проект после выполнения ЛР3. Проведя сравнение, монжо выделить следующее:

\begin{itemize}
    \item использование модели <<водопада>> показало (дата окончания проекта 23.08), что проект будет завершен позже, чем при изначальном подходе (18.07);
    \item при <<водопаде>> затраты меньше (38 916 руб), чем при изначальном подходе (47 340 руб (без учета совещаний)).
\end{itemize}

\imgs{task3-decomp}{h!}{0.4}{Декомпозиция ЛР2}
\imgs{task3-lab3}{h!}{0.4}{Результат ЛР3}
