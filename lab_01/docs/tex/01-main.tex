% \maketableofcontents

\chapter{Выполнение лабораторной работы}


\section{Цель работы}

Целью лабораторной работы №1 является освоение возможностей программы \textbf{Microsoft Project} для планирования проекта по разработке программного обеспечения.



\section{Тренировочное задание}

Осуществить планирование проекта с временными характеристиками, которые представлены на рисунке \ref{img:test-task}. По умолчанию используются фиксированный объем ресурсов.

\imgs{test-task}{h!}{0.5}{Временные характеристики тестового задания}

Дата начала проекта --- 1-ый рабочий день марта текущего года. Провести планирование работ проекта, учитывая следующие связи между задачами:

\begin{enumerate}
    \item Предусмотреть, что C, J и D – исходные работы проекта, которые можно
    начинать одновременно.
    \item Работа A следует за D, а работа I – за A.
    \item Работа H следует за I.
    \item Работа F следует за H, но не может начаться, пока не завершена С.
    \item Работа G следует за I.
    \item Работа E следует за J.
    \item Работа B следует за E.
\end{enumerate}


\subsection{Выполнение задания}

В проекте была установлена дата его начала на 1 марта (1-ый рабочий день марта 2023 года). Затем был создан список задач и установлены связи между задачами в соответствии с заданием. Результат представлен на рисунке \ref{img:test-result}.

% \imgs{test-result}{h!}{0.5}{Результат выполнения тестового задания}


\section{Краткое описание проекта}

Команда разработчиков из 16 человек занимается созданием карты города на основе собственного модуля отображения. Проект должен быть завершен в течение 6 месяцев. Бюджет проекта: 50 000 рублей.


\section{Задание 1}

Настроить рабочую среду проекта.

\subsection{Выполнение задания}

\begin{enumerate}
    \item Во вкладке <<Сведения о проекте>> установлена дата начала проекта и стандартный календарь (рисунок \ref{img:task1-1}).
    \imgs{task1-1}{h!}{0.5}{Сведения о проекте}

    \item В <<Параметрах>> во вкладке <<Расписание>> установлена длительность работ в неделях, объем работ в часах, тип работ по умолчанию, трудозатраты фиксированные. При этом количество рабочих часов в день равно 8, а в неделю -- 40. Также установлено начало рабочей недели на понедельник, а финансового года -- на явнварь. Время начала рабочего дня установлено на 9:00, а окончание на 18:00 (рисунок \ref{img:task1-2}).
    \imgs{task1-2}{h!}{0.5}{Параметры проекта}

    \item Во вкладке <<Изменить рабочее время>> установлены праздники и выходные дни (рисунок \ref{img:task1-3}).
    \imgs{task1-3}{h!}{0.5}{Праздничные дни}

    \item Выведена суммарная задача проекта и записана основная информация о проекте в <<Заметки>> (рисунок \ref{img:task1-4}).
    \imgs{task1-4}{h!}{0.5}{Суммарная задача проекта и <<Заметки>>}
\end{enumerate}


\section{Задание 2}

Создать список задач.

\subsection{Выполнение задания}

В соответствии с заданием введены задачи в проект (рисунок \ref{img:task2}).

\imgs{task2}{h!}{0.3}{Список задач}


\section{Задание 3}

Структурировать список задач проекта.

\subsection{Выполнение задания}

В соответствии с заданием установлены уровни списка задач в проекте (рисунок \ref{img:task3}).

\imgs{task3}{h!}{0.3}{Структура задач}


\section{Задание 4}

Установить связи между задачами проекта.

\subsection{Выполнение задания}

В соответствии с заданием установлены связи между задачами в проекте (рисунок \ref{img:task4}).

\imgs{task4}{h!}{0.3}{Связи задач}


\section*{Вывод}

В результате выполнения лабораторной работы было определено, что проект должен завершитья 15.09.23, что на 15 дней больше ожидаемого срока выполнения проекта, который был установлен на отметке 6 месяцев.
